\documentclass[handout,fleqn, t]{beamer}

\usepackage{caption}
\captionsetup[figure]{name=}

\usepackage{listings}
\lstset{numbers=left,frame=single,rulesep=0em}

\usepackage{pgfpages}
\pgfpagesuselayout{2 on 1}[letterpaper,border shrink=5mm]
\pgfpageslogicalpageoptions{1}{border code=\pgfusepath{stroke}}
\pgfpageslogicalpageoptions{2}{border code=\pgfusepath{stroke}}
\pgfpageslogicalpageoptions{3}{border code=\pgfusepath{stroke}}
\pgfpageslogicalpageoptions{4}{border code=\pgfusepath{stroke}}

\title{What Does This Program Do?\linebreak (Looping \& Arrays)}
\subtitle{GBW ACSL - Contest \#3}
\author{Clayton O'Neill --- clayton@oneill.net}
\date{2014-2015}

\setbeamertemplate{footline}[frame number]

\setlength{\parskip}{1em plus 1pt minus 1pt}

\begin{document}
\frame{\titlepage}

\begin{frame}[fragile]
  \frametitle{How Do Loops Work?}
  Loops in BASIC have the following form:

  \begin{lstlisting}[gobble=4]
    FOR X = 0 TO 30 STEP 10
      PRINT X
    NEXT X
  \end{lstlisting}

  This program will print 0, 10, 20, 30.  The loop starts on line 1.  The
  \lstinline{FOR X} part specifies that X is the variable that will change
  each time the loop is run.

  The \lstinline{0 TO 30} part of line 1 sets the initial value of X when the
  loop starts, and specifies when the loop will end.

  The last part part of line 1, \lstinline{STEP 10}, specifies how much will be
  added to X every time the loop ends.  The \lstinline{STEP} is optional and
  defaults to 1.

  Finally, on line 3 we see \lstinline{NEXT X}.  This marks the end of the loop
  where X is the loop variable.  This is important for nested loops.

\end{frame}

\begin{frame}[fragile]
  \frametitle{Nested Loops}

  This is an example an outer and inner loop, or \textit{nested} loops.

  \begin{lstlisting}[gobble=4]
    FOR X = 1 TO 3
      FOR Y = 1 TO 3
        PRINT X,Y
      NEXT Y
    NEXT X
  \end{lstlisting}

  The output will be:
    \begin{lstlisting}[gobble=4]
    1 1
    1 2
    1 3
    2 1
    ...
    3 3
    \end{lstlisting}

\end{frame}

\begin{frame}[fragile]
  \frametitle{Backwards Loops}

  The \lstinline{STEP} option can also be negative to loop backwards:

  \begin{lstlisting}[gobble=4]
    FOR X = 30 TO 0 STEP -10
      PRINT X
    NEXT X
  \end{lstlisting}

  This will print:
  \begin{lstlisting}[gobble=4]
    30
    20
    10
    0
  \end{lstlisting}
\end{frame}

\begin{frame}[fragile]
  \frametitle{Decimal Numbers as \lstinline{STEP}}

  The \lstinline{STEP} can take non-whole, or decimal numbers as shown below:

  \begin{lstlisting}[gobble=4]
    FOR X = 30 TO 0 STEP -7.5
      PRINT X
    NEXT X
  \end{lstlisting}

  This will print:
  \begin{lstlisting}[gobble=4]
    30
    20.5
    11
    1.5
  \end{lstlisting}

  How does the loop decide to end in this case?
\end{frame}


\begin{frame}
  \frametitle{Technical Implementation}

  When a For...Next loop starts, Visual Basic evaluates start, end, and step.
  Visual Basic evaluates these values only at this time and then assigns start
  to counter. Before the statement block runs, Visual Basic compares counter to
  end. If counter is already larger than the end value (or smaller if step is
  negative), the For loop ends and control passes to the statement that follows
  the Next statement. Otherwise, the statement block runs.

  Each time Visual Basic encounters the Next statement, it increments counter
  by step and returns to the For statement. Again it compares counter to end,
  and again it either runs the block or exits the loop, depending on the
  result. This process continues until counter passes end or an Exit For
  statement is encountered.

  The loop doesn't stop until counter has passed
  end...\footnote{https://msdn.microsoft.com/en-us/library/5z06z1kb.aspx}
\end{frame}

\begin{frame}[fragile]
  \frametitle{No Nothing Loops}

  Loops can result in \textbf{nothing} being done!

  \begin{lstlisting}[gobble=4]
    FOR X = 30 TO 0
      PRINT X
    NEXT X
  \end{lstlisting}

  This will print out nothing, because X will be assigned 30, which is greater
  than 0, so the loop doesn't even run once.
\end{frame}

\begin{frame}[fragile]
  \frametitle{Common Pattern: Checking If Evenly Divisible}

  One pattern that is common in ACSL question is shown below:

  \begin{lstlisting}[gobble=4]
    FOR X = 1 TO 20
      IF INT(X/2) = X/2 THEN PRINT X
    NEXT X
  \end{lstlisting}

  This program prints all numbers between 1 and 20 that are divisible by two.
  It does this by dividing the number by 2 and rounding it it towards zero,
  then comparing that to just the number divided by 2.  If those are the same,
  that means the number divided by two has no remainder.

  For example, if X were 3, then the first part of the comparison on line 2
  would be, 1, but the second part would be 1.5.  Those two are are not equal,
  so the 3 isn't evenly divisible by 2.
\end{frame}

\frame {
  \frametitle{What Does This Program Do: Arrays}
  \begin{itemize}
    \item Arrays are variables that hold more than one value.
    \item Another word commonly used for arrays is \textit{matrix}.
    \item One way to think about them is as a table like we saw with the
      adjacency matrix.
    \item In ACSL you will normally see arrays that either have one or two
      \textit{dimensions}.
    \item The number of dimensions is determines how many variables you use to
      access the array.
    \item One dimension arrays are accessed, or indexed, with one variable.
    \item Two dimension arrays are accessed, or indexed, with two variables.
    \item You'll commonly see arrays paired with loops.
  \end{itemize}
}

\begin{frame}[fragile]
  \frametitle{One Dimensional Array Example}

  The example below initializes the array A using the \lstinline{READ} command
  then counts all array elements that are greater than 2 and prints it out.

  \begin{lstlisting}[gobble=4]
    DATA 3, 3, 2, 1, 2, 4, 7
    FOR X = 1 to 7
      READ Y
      A(X) = Y
    NEXT X
    C = 0
    FOR X = 1 TO 7
      IF A(X) > 2 C = C + 1
    NEXT X
    PRINT C
  \end{lstlisting}
\end{frame}

\begin{frame}[fragile]
  \frametitle{Two Dimensional Array Example}

  Assume that the array A has the values given in the table below.  What would
  this program print?

  \begin{lstlisting}[gobble=4]
    FOR X = 1 TO 4
      C = 0
      FOR Y = 1 TO 3
        IF INT(A(X,Y)/2) = A(X,Y)/2 THEN C = C + 1
      NEXT Y
      PRINT C
    NEXT X
  \end{lstlisting}
  \begin{tabular}{|c|c|c|c|}
    \hline 37 & 14 & 10 & 2 \\
    \hline 2 & 3 & 11 & 6 \\
    \hline 21 & 0 & 1 & 8 \\
    \hline
  \end{tabular}

  Bottom left corner is 1,1 (21).
\end{frame}

\begin{frame}
  \frametitle{Miscellanous Hints and Tips}
  \begin{itemize}
    \item In ACSL problems arrays are sometimes declared using the
      \lstinline{DIM} declaration.  You can ignore these.  In ACSL BASIC, declaring arrays ahead of time isn't necessary.
    \item ACSL array problems frequently take the form: Initialize array/Change
      Array/Summarize Array.
    \item When you are given a two dimensional array in a problem, be very
      careful about where the two axes start.
  \end{itemize}
\end{frame}

\end{document}
