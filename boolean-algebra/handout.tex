\documentclass[handout,fleqn, t]{beamer}

\usepackage{color, colortbl}
\definecolor{Gray}{gray}{0.9}

\usepackage{pgfpages}
\pgfpagesuselayout{2 on 1}[letterpaper,border shrink=5mm]
\pgfpageslogicalpageoptions{1}{border code=\pgfusepath{stroke}}
\pgfpageslogicalpageoptions{2}{border code=\pgfusepath{stroke}}
\pgfpageslogicalpageoptions{3}{border code=\pgfusepath{stroke}}
\pgfpageslogicalpageoptions{4}{border code=\pgfusepath{stroke}}

\title{Boolean Algebra}
\subtitle{GBW ACSL - Contest \#3}
\author{Clayton O'Neill --- clayton@oneill.net}
\date{2014-2015}

\setbeamertemplate{footline}[frame number]

\begin{document}

\frame{\titlepage}

\frame {
  \frametitle{What is Boolean Algebra?}
  \begin{itemize}
    \item Boolean algebra is algebra using only two numbers: 0 and 1
    \item Closely related to Digital Electronics in contest \#4
    \item It uses the logic operations we've already learned: AND, OR, XOR, NOT
    \item NOT is represented with a line over the value: \( \overline{A} \)
    \item AND is represented by multiplication (A * B) or AB
    \item XOR is represented with \( \oplus{} \) : \( A \oplus{} B \)
    \item OR is represented by addition: A + B
    \item Order of operation NOT first then AND, XOR and OR.
    \item 0 is considered FALSE, 1 is considered TRUE
  \end{itemize}
}

\frame {
  \frametitle{Truth Tables for Boolean Operations}

  \begin{tabular}{ c c | c | c | c | c}
    A & B & \(\overline{A}\) & A + B & A * B & \(A \oplus{} B \) \\
    \hline
    0 & 0 & 1 & 0 & 0 & 0 \\
    0 & 1 & 1 & 1 & 0 & 1 \\
    1 & 0 & 0 & 1 & 0 & 1 \\
    1 & 1 & 0 & 1 & 1 & 0
  \end{tabular}

  \vspace{1em}
  Remember:
  \begin{itemize}
    \item + is short for OR
    \item * is the short way of saying AND
    \item \(\oplus{}\) is short for Exclusive-OR (XOR)
    \item \(\overline{A}\) is short for NOT A
  \end{itemize}
  \vspace{1em}
  Don't confuse + and * in boolean algebra with normal math.
  Remember that \(1 + 1 = 1\) in boolean algebra because it's
  another way of writing 1 OR 1 = 1.  You need to memorize
  truth tables to solve these.
}

\frame {
  \frametitle{Translating Boolean Algebra to Logic}

  Here are a few examples of how you can mentally translate boolean algebra
  expressions into logic statements you may be more used to.


  \vspace{1em}
  A+B = A OR B

  \vspace{1em}
  AB = A * B = A AND B

  \vspace{1em}
  \(A \oplus{} B = \) A XOR B

  \vspace{1em}
  \(A \overline{B} = A * \overline{B} \) = A AND (NOT B)

  \vspace{1em}
  \( (\overline{A+B})BC \)
  \( = (\overline{A+B}) * B * C \)
  = NOT(A OR B) AND B AND C


}

\frame {
  \frametitle{Boolean Algebra Identities: The Rules}

  Boolean algebra has rules you can apply to solve or simplify equations.  You
  will need to memorize these and practice using them.  The numbers of the
  identities shown here are from the ACSL materials.

  \vspace{1em}
  \begin{enumerate}
    \item Communicative Property --- Order of OR (+) doesn't matter:
      \[ A+B = B+A \]
    \item Associative Property --- Order of AND (*) doesn't matter:
      \[ A*B = B*A \]
    \item Distributive Property:
      \[ A * (B + C) = A * B + A * C \]
  \end{enumerate}
}

\frame {
  \frametitle{De Morgan's Law\footnote[1]{They are named after Augustus De
  Morgan, a 19th-century British mathematician}}

  \begin{enumerate}
    \setcounter{enumi}{3}
    \item Negated OR of two items is the same as AND of the negatives.
      \[ \overline{A + B} = \overline{A} * \overline{B} \]
    \item Negated AND of two items is the same as OR of the negatives.
      \[ \overline{A * B} = \overline{A} + \overline{B} \]
  \end{enumerate}
}
\frame {
  \frametitle{The Basics - OR}
  \begin{enumerate}
    \setcounter{enumi}{5}
    \item Value ORed with zero(FALSE) is itself.
      \[ A + 0 = A \]
    \item Value ORed with one(TRUE) is always one(TRUE).
      \[ A + 1 = 1 \]
    \item Value ORed with it's opposite is always one(TRUE).
      \[ A + \overline{A} = 1 \]
    \item Value ORed with itself is one(TRUE).
      \[ A + A =  1 \]
  \end{enumerate}
}
\frame {
  \frametitle{The Basics - AND}
  \begin{enumerate}
    \setcounter{enumi}{5}
    \item Value ANDed with zero(FALSE) is always zero(FALSE).
      \[ A * 0 = 0 \]
    \item Value ANDed with one(TRUE) is itself.
      \[ A * 1 = A \]
    \item Value ANDed with it's opposite is always zero(FALSE).
      \[ A * \overline{A} = 0 \]
    \item Value ANDed with itself is itself.
      \[ A * A =  A \]
  \end{enumerate}
}
\frame {
  \frametitle{Identities: Part V}
  \begin{enumerate}
    \setcounter{enumi}{10}
    \item \( A + \overline{A} * B = A + B \)
    \item \((A + B) * (A + C) = A + B * C \)
    \item \( (A + B) * (C + D) = A*C + A*D + B*C + B*D \)

      This is an expanded form of the more general Distributive Property in \#3
    \item \(A * (A + B) = A \)

      If A is true then it doesn't matter if B is true or false, and if A is
      false, then the first A will force the AND to be false.
    \item \( A \oplus{} B = A * \overline{B} + \overline{A} * B \)
    \item \( \overline{A \oplus{} B} = \overline{A} \oplus{} B = A \oplus{} \overline{B} \)
  \end{enumerate}
}
\frame {
  \frametitle{What do you need to know?}
  There are mainly two types of problems on the ACSL practice tests:

  \begin{itemize}
    \item Simplify Expression \\
    These require you to use the identities above to make the problem as simple
    as possible.  You will need to memorize most or all of the identities and
    know how to apply them.

    \item Simplify Expression and Solve \\
     These are the same as above, require you also find all the values that can
     make the expression true but can be easier in some cases.  You can stop
     simplifying when you feel the problem can be solved with a truth table,
     meaning you can get away not knowing the identities quite as well for
     these problems.
   \end{itemize}

  These aren't the the only problem types, just the most common on past tests.
}
\frame[allowframebreaks] {
  \frametitle{Simplify This Expression Example}
  \enumerate {
    \item Simplify:
      \[ (\overline{A} * B + A*B)(\overline{A*B}) \]
    \item Apply Distributive Property (\#3) to move B outside of first parentheses:
      \[ B * (\overline{A} + A) * (\overline{AB}) \]
    \item Apply De Morgan's Law (\#5) to the second parentheses:
      \[ B * (\overline{A} + A) * (\overline{A} + \overline{B}) \]
    \item Apply identity \#8 to simplify first parentheses:
      \[ B * 1 * (\overline{A} + \overline{B}) \]
    \item Apply identity \#7 to remove 1:
      \[ B * (\overline{A} + \overline{B}) \]
    \item Apply Distributive Property (\#3) to spread B:
      \[ B * \overline{A} + B * \overline{B} \]
    \item Apply identity \#8 to convert second half to zero:
      \[ B * \overline{A} + 0 \]
    \item Apply identity \#6 to get rid of zero:
      \[ B * \overline{A} \]
  }
}

\frame[allowframebreaks] {
  \frametitle { Simplify and Solve Example}
  \begin{enumerate}
    \item Simplify first:
      \[ AB(\overline{A} + C) + \overline{A}(B + C) \]
    \item Apply Distributive identity (\#3) to both sides:
      \[ AB\overline{A} + ABC + \overline{A}B + \overline{A}C \]
    \item Apply identity \#8 to convert \(A\overline{A}\) to zero:
      \[ 0 * B + ABC + \overline{A}B + \overline{A}C \]
    \item Apply identity \#6 to remove \( 0 * B \):
      \[ ABC + \overline{A}B + \overline{A}C \]
  \end{enumerate}

  At this point no further simplification is possible, so we must figure out
  what values for A, B and C will make the expression true.  We want this truth
  table to contain each of the parts of the expression that we'll solve for:
}

\frame {
  \frametitle{Solving the Equation I}
  To figure out what values will make the expression true, we make a truth
  table that shows every value for A, B and C.

  \vspace{1em}
  \begin{tabular}{ c c c }
    A & B & C \\
    \hline
    0 & 0 & 0 \\
    0 & 0 & 1 \\
    0 & 1 & 0 \\
    0 & 1 & 1 \\
    1 & 0 & 0 \\
    1 & 0 & 1 \\
    1 & 1 & 0 \\
    1 & 1 & 1
  \end{tabular}
}

\frame {
  \frametitle{Solving the Equation II}
  Then we want to add columns for each of the parts of the expression, and the
  expression itself, and then solve each part:

  \vspace{1em}
  \begin{tabular}{ c c c| c c c | c }
    A & B & C & ABC & \(\overline{A}B\) & \(\overline{A}C\) & \(ABC + \overline{A}B + \overline{A}C \)\\
    \hline
    0 & 0 & 0 & 0 & 0 & 0 & 0 \\
    \rowcolor{Gray}
    0 & 0 & 1 & 0 & 0 & 1 & 1 \\
    \rowcolor{Gray}
    0 & 1 & 0 & 0 & 1 & 0 & 1 \\
    \rowcolor{Gray}
    0 & 1 & 1 & 0 & 1 & 1 & 1 \\
    1 & 0 & 0 & 0 & 0 & 0 & 0 \\
    1 & 0 & 1 & 0 & 0 & 0 & 0 \\
    1 & 1 & 0 & 0 & 0 & 0 & 0 \\
    \rowcolor{Gray}
    1 & 1 & 1 & 1 & 0 & 0 & 1
  \end{tabular}
\vspace{1em}

  The rows in gray are the ones the expression is true for.  ACSL expects you
  to provide them in alphabetic order as a sequence of numbers.  For example:
  (0, 0, 1), (0, 1, 0), (0, 1, 1), and (1, 1 1).
}

\frame {
  \frametitle{Tips and Tricks}
  \itemize {
    \item Be careful about NOT:
      \[ \overline{A + B} \neq \overline{A} + \overline{B} \]
      \[ \overline{A * B} \neq \overline{A} * \overline{B} \]
    \item When simplifying look for opportunities to combine negated terms to
      eliminate them.  (\( B * \overline{B} = 0\)).
    \item Logical reasoning can save you a lot of time.  A truth table for
      \(A(BC + AD)\) must show A always being true because A is being
      ANDed with the larger expression

    \item When working the problems, be \emph{very} careful about copying the
      bars above expressions.
    \item Remember that AND is implied sometimes.  For example \(AB(A + D) = A
      * B * (A + D)\).
    \item In ACSL examples the the Communicative Property is frequently used to
      reorder the expressions.  For example, they may prefer
      \(A\overline{B}C\) instead of \(CA\overline{B}\).
  }
}
\end{document}
